\documentclass{article}

\usepackage{graphicx}
\usepackage[utf8]{inputenc}
\usepackage{pmboxdraw}
\usepackage[titletoc,title]{appendix}
\usepackage{placeins}

\author{Nathaniel Knight}
\date{\today}

\title{SHARED Contributor Guide for Clinical Data}

\begin{document}

\maketitle

\begin{abstract}
  This document has information about the format for
  \textbf{submitting clinical data} to the SHARED collaboration and
  instructions on how to prepare, check, encrypt, and securely submit
  your data.
\end{abstract}

\newpage

\section{Using the data submission tool}

With the \verb|squishd.py| utility you'll be able to 

\begin{itemize}
\item Check that your data is in the format we're expecting, and
  highlight any differences.
\item Encrypt your data with the SHARED project's public key.
\item Securely transmit it to the the SHARED site in Vancouver to be merged into the database.
\end{itemize}

The tool isn't ready yet, but we expect to distribute it sometime in June 2017.


\section{Data format}

Before data can be submitted to the SHARED merge process it needs to
be in a standard format: \textbf{CSV files} for participant clinical
and behavioral data, and \textbf{FASTA files} for sequence data. The
files it expects are listed in Table \ref{tbl:expectedfiles}, and
described in detail below. The relationships between files is shown in
Figure \ref{fig:relationships}. A read-to-check data-prep directory is
shown in Figure \ref{fig:data-prep-dir}.

The fields (and the data they can contain) in each file are described
in the tables below.

\begin{description}
\item[number] A whole or decimal number. E.g: 1, 7.4, -12
\item[text] A string of characters. If you need to include commas
  (e.g. in free-form notes) you must enclose the field in
  double-quotes. If you're using a program like Excel or a
  CSV-specific software (like \texttt{csv} in the Python standard
  library) this may be done automatically. E.g: 1a, sufosbuvir, ``The
  truth, the whole truth, and nothing but the truth.''.
\item[date] A date in the format YYYY-MM-DD. E.g: 2017-05-17, 1969-07-20, 1980-05-08.
\end{description}

Fields that are \textbf{required} are marked with a bullet (•) in the
tables below.  Many fields can be left blank out, indicating that the
data wasn't collected or is unknown (they'll be represented as
\texttt{NULL} in the database).

The clinical data file in particular has many fields; we don't expect
that anyone will use them all. If you don't collect a particular kind
of data, you should ommit the column.


\begin{table}[h!]
  \caption{SHARED submission data format}
  \label{tbl:expectedfiles}
  \begin{tabular}{rp{8cm}}
    File Name (case insensitive) & Contents \\ \hline
    \verb|partcipants.csv| & Participant ID codes and demographic data. \\
    \verb|behavior.csv| &  Participants' behavioral risk factors. \\
    \verb|clinical.csv| & Participants' test results and relevant clinical history. \\
    \verb|ltfu.csv| & Records of loss-to-follow-up. \\
    \verb|treatment.csv| &  Participants' treatment information. \\
    \verb|isolates.csv| & Participants' viral isolate data (sequences stored separately). \\
  \end{tabular}
\end{table}

\begin{figure}
  \caption{Relationships between SHARED submission data files}
  \label{fig:relationships}
  \includegraphics[width=\textwidth]{tmp/clinical_relationships}
\end{figure}

The CSV files you provide should have \textbf{headers} containing the
field name in each column, but the \textbf{order} of the columns
doesn't matter, and any extra columns will be ignored (so you can add
notes, references, etc.). You can also \textbf{omit} the columns for
fields that aren't required if they don't apply to your study. The
\textbf{case} of the filenames and header names also doesn't matter
(they will all be converted to lower-case before they're checked and
loaded).

Sequence data is expected in a \textbf{sub-directory} called
\texttt{sequences}, with file-names in the \texttt{isolates.csv}
file. The names in the examples are just examples; you can use
whatever naming convention is easiest for you, as long as the
filenames in \texttt{isolates.csv} match the files in the
\texttt{sequences/} directory.

\begin{figure}
  \caption{Layout of a SHARED submission data directory}
  \label{fig:data-prep-dir}
\begin{verbatim}
 my_data_prep_directory/
 ├── behavior.csv
 ├── clinical.csv
 ├── isolates.csv
 ├── ltfu.csv
 ├── partcipants.csv
 ├── treatment.csv
 └── sequences/
     ├── seq1.fasta
     ├── seq2.fasta
     etc.
\end{verbatim}
\end{figure}


\begin{table}[h!]
  \centering
  \caption{The \texttt{participants.csv} file's fields}
  \label{tbl:participants.csv}
  \begin{tabular}{cllp{6cm}p{4cm}}
    R & Name            & Type      & Description & Possible Values\\ \hline
    • & person          & text    &
      The \textbf{anonymous ID} used for the participant in this
      merge. It will be referred to in other CSV files to associate
      other kinds of data to this participant. The exact format of
      this field isn't particularly important as long as the keys in
      the other files \textbf{match}; use whatever works best for you
      (numbers, UUIDs, etc.). Note that keys are compared as
      \textbf{text}, so $"01" \neq "1"$ and $"1" \neq "1.0"$. 
      \\
      & country         & text    & The country where the person participated in the study. \\
      & city            & text    & The city where the participant participated in the study. \\
      & sex             & text    & The participant's sex at birth.  & male, female, other \\
      & year\_of\_birth & integer   & The participant's year of birth. \\
      & ethnicity       & text    & The participant's ethnicity. & am-nat, asian, black, hisp, pa-isl, white \\
  \end{tabular}
\end{table}


\begin{table}[h!]
  \centering
  \caption{The \texttt{behavior.csv} file's fields}
  \label{tbl:behavior.csv}
  \begin{tabular}{cllp{6cm}p{4cm}}
    R & Name        & Type          & Description & Possible Values\\ \hline
    • & person      & text        &
    The key to a person in the \texttt{participants.csv}
    file. \texttt{squishd.py} will check that all the participants in
    this file are present in the participants file. \\
      & date\_collected & date      & The date this data was collected. \\
      & sex\_ori        & text    & The participant's sexual orientation & heterosexual, homosexual, bisexual, other \\
      & idu             & bool      & Has the participant ever used injection drugs? \\
      & ndu             & bool      & Has the participant ever used non-injection drugs? \\
      & idu\_recent     & bool      & Has the participant used injection drugs in the past 6 months? \\
      & ndu\_recent     & bool      & Has the participant used non-injection drugs in the past 6 months? \\    
      & prison          & bool      & Has the participant ever been to prison? \\
  \end{tabular}
\end{table}


\begin{table}[h!]
  \centering
  \caption{The \texttt{clinical.csv} file's fields}
  \label{tbl:clinical.csv}
  \begin{tabular}{cllp{6cm}p{4cm}}
    R & Name        & Type      & Description & Possible Values\\ \hline
  • & person      & text    & The key to a person in the \texttt{participants.csv}
    file. \texttt{squishd.py} will check that all of the participants in this file
    are present in the participants file. \\
    & hiv           & bool      & Is the participant co-infected with HIV? & \\
    & hbv           & bool      & Is the participant co-infected with HBV? & \\
    & ost           & bool      & Has the participant undergone opioid substitution therapy in the last six months? &  \\
    & chir          & bool      & Does the participant have cirrhosis? & \\
    & metavir       & text      & Metavir score. & A0, A1, A2, A3, F0, F1, F2, F3, F4 \\
    & metavir\_by   & text      & Method used to determine metavir score. & Fibroscan, Biopsy, Clinical, Other \\
    & stiff         & number    & Liver stiffness (in kPa). & \\
    & alt           & number    & Alanine aminotransferase level (in U/L). & \\
    & ast           & number    & Aspartate aminotransferase level (in U/L). & \\
    & crt           & number    & Creatinine level (in $\mu$M/L). & \\
    & egfr          & number    & Estimated glomerular filtration rate (in mL/minBSAc). & \\
    & ctp           & number    & Child-Turcotte-Pugh score. & \\
    & meld          & number    & MELD score. & \\
    & ishak         & number    & Ishak score. & \\
    & bil           & number    & Bilirubin test result, in mg/dL. & \\
    & hemo          & number    & Hemoglobin test result, in g/dL. & \\
    & alb           & number    & Albumin test result, in g/dL. & \\
    & inr           & number    & International Normalized Ratio test result. & \\
    & phos          & number    & Phosphate test result, in mg/dL. & \\
    & urea          & number    & Urea test result, in ng/dL. & \\
    & plate         & number    & Platelet count, in cells/mm$^3$. & \\
    & CD4           & number    & CD4 count, in cells/mm$^3$. & \\
    & crp           & number    & C-Reactive Protein test result, in mg/L. & \\
    & il28b         & string    & The participant's IL28B-rs12979860 genotype. & TT, TC, CC \\
    & asc           & bool      & Did the participant have ascites before or during treatment? & \\
    & var\_bleed    & bool      & Did the participant have variceal bleeding before or during treatment? & \\
    & hep\_car      & bool      & Did the participant have hepatocellular carcinoma before or during treatment? & \\
    & transpl       & bool      & Has the participant had a liver transplant? & \\
  \end{tabular}
\end{table}


\begin{table}[h!]
  \centering
  \caption{The \texttt{ltfu.csv} file's fields}
  \label{tbl:ltfu.csv}
  \begin{tabular}{cllp{6cm}p{4cm}}
    R & Name            & Type      & Description & Possible Values \\ \hline
    • & person          & text      & The key to a person in the \texttt{participants.csv}
    file. \texttt{squishd.py} will check that all of the participants in this file
    are present in the participants file. & \\
      & ltfu\_dt        & date      & The date the participant was lost to follow up. & \\
      & ltfu\_reason    & text      & The reason the participant was lost to follow up (if applicable). & \\
      & died            & bool      & Is the participant deceased? & \\
      & cod             & text    & Cause of death (if applicable). & See Appendix \ref{apx:cod}.\\
  \end{tabular}
\end{table}


\begin{table}[h!]
  \centering
  \caption{The \texttt{treatment.csv} file's fields}
  \label{tbl:treatment.csv}
  \begin{tabular}{cllp{6cm}p{4cm}}
    R & Name                & Type      & Description   & Possible Values\\ \hline
    • & person              & text    & The key to a person in the \texttt{participants.csv}
    file. \texttt{squishd.py} will check that all of the participants in this file
    are present in the participants file. & \\
      & first\_treatment    & bool      & Is this the participant's first treatment? & \\
      & start\_dt           & date      & The treatment start date & \\
      & end\_dt\_sch        & date      & Scheduled treatment end date & \\
      & end\_dt\_act        & date      & Actual treatment end date (if different from scheduled). \\
      & end\_dt\_bound      & text    &
    Uncertainty on end\_dt\_bound (default is $=$). & $=, <, >$ \\
      & pr                  & bool      & Has the participant been treated with pegylated interferon with or without ribavirin before? & \\
      & daa1                & bool      & Has the participant been treated with first-line direct-acting antiviral drugs before? & \\
      & daa2                & bool      & Has the participant been treated with second-line direct-acting antiviral drugs before? & \\
      & regimen             & text      & What drug regimen was the participant taking? & See Appendix \ref{apx:drug-regimens} \\
      & prev\_regimen       & text      & If the participant has been treated before, what treatment regimen were they take previously?  & See appendix \ref{apx:drug-regimens}. \\
      & pprev\_regimen      & text      & If the participant has had two previous treatments, what regimen were they taking before-last? & \ref{apx:drug-regimens}. \\
      & response            & text      & How did the participant respond to treatment? & See Appendix \ref{apx:response} \\
      & notes               & text    & Additional notes (if required). & \\
  \end{tabular}
\end{table}


\begin{table}[h!]
  \centering
  \caption{The \texttt{isolates.csv} file's fields}
  \label{tbl:isolates.csv}
  \begin{tabular}{cllp{6cm}p{4cm}}
    R & Name        & Type      & Description & Possible Values\\ \hline
    • & person      & text    & The key to a person in the
    \texttt{participants.csv} file, indicating who this isolate came
    from. \texttt{squishd.py} will check that all of the participants
    in this file are present in the participants file. & \\
    • & seq\_file   & text    & The name of a FASTA file in the
    \texttt{sequences} folder containing the concensus sequence data for this viral isolate. & \\ 
    • & method      & text    & The sequencing method used. & sanger, ngs \\
    • & cutoff      & number  & The cutoff value used to create the concensus sequence, as a percentage. & \\
    • & kind        & text    & Whether the sequence is from a baseline sample, an end-of-treatment sample, or a follow-up sample 4, 12, or 24 weeks after treatment. & bl, eot, fw4, fw12, fw24 \\
    & genotype    & text    & The genotype of this viral isolate. &
    1, 2, 3, 4, 5, 6, mixed, indeterminate, recombinant \\
    & subgenotype & text    & The sub-genotype of this viral isolate. & \\
    & strain      & text    & The strain of this viral isolate. & \\
    & isln\_dt    & date      & The date this virus was isolated. & \\
    
  \end{tabular}
\end{table}


\section{Going deeper}

This guide describes the simplest way to load data into SHARED, but
the database can store more detailed information. It can also change
in future to accommodate useful data that we haven't foreseen. More
detailed data formats are described below. For more detailed schema
information or to request additions to the schema, contact Project
Coordinator Anita Howe (\texttt{ahowe@cfenet.ubc.ca}) or Software
Developer Nat Knight (\texttt{nknight@cfenet.ubc.ca}).


\FloatBarrier
\begin{appendices}

\newpage
\section{Detailed drug regimen information}
\label{apx:drug-regimens}

Incomplete.

We're currently getting this information from the virology experts to
the publishing person. Updates will be forthcoming! Because different
regimens are approved in different jurisdictions, we expect to have to
update this table fairly frequently as we merge our first few
datasets.


\newpage
\section{Cause of Death codes}
\label{apx:cod}
\begin{table}[h]
  \caption{Coding scheme for Cause of Death}
  \label{tbl:cause-of-death}
  \begin{tabular}{rl}
    Code                & Cause \\ \hline
    \texttt{liv}        & Liver disease \\
    \texttt{aid}        & AIDS related \\
    \texttt{odo}        & Overdose \\
    \texttt{can}        & Cancer \\
    \texttt{cir}        & Circulatory \\
    \texttt{res}        & Respiratory \\
    \texttt{dia}        & Diabetes \\
    \texttt{gen}        & Genitourinary \\
    \texttt{tra}        & Injuries/Trauma \\
    \texttt{cer}        & Cerebral/Neurologic \\
    \texttt{dig}        & Digestive (extrahipatic) \\
    \texttt{oth}        & Other \\
  \end{tabular}
\end{table}



\newpage
\section{Treatment Response Codes}
\label{apx:response}
\begin{table}[h]
  \caption{Coding scheme for Treatment Response}
  \label{tbl:cause-of-death}
  \begin{tabular}{rl}
    Code                & Cause \\ \hline
    \texttt{nr}         & No-response. The participant had no more than a 2-log reduction in viral load. \\
    \texttt{bkt}        & Breakthrough. The participant had an on-treatment relapse. \\
    \texttt{rl}         & Relapse. The participant relapsed after treatment ended. \\
    \texttt{svr12}      & Sustained viral response. The participant had no detectable viral load at 12 weeks. \\
    \texttt{svr24}      & Sustained viral response. The participant had no detectable viral load at 24 weeks. \\
  \end{tabular}
\end{table}


\end{appendices}


\end{document}
%% TODO(nknight): references
% - https://academic.oup.com/cid/article/58/8/1055/357542/Mortality-Among-Persons-in-Care-With-Hepatitis-C
% - Jason's sheet
% etc.
